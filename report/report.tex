\documentclass[a4paper,11pt]{article}
\usepackage[T1]{fontenc}
\usepackage[utf8]{inputenc}
\usepackage{lmodern}
\usepackage{url}

\title{
Mining of Massive Dataset\\
Project 2 Report
}
\author{Hao Jiteng, Zhou Lizhi, Yang Fangzhou}

\begin{document}

\maketitle
%\tableofcontents

\begin{abstract}
K-means is a simple yet useful clustering algorithm. It's underlying natural
implies that it could be parallelized or distributized. Some implementation of
kmeans employs platforms such as Hadoop and CUDA to boost the process of mining
of massive dataset.
In our project we implement k-means algorithm on Apache Hadoop Project. We ran
our algorithm on our tiny cluster. Evaluation has been done to measure our
algorithm.
\end{abstract}

\section{Introduction}
\subsection{Hadoop}
Following is the introduction of Hadoop on its project
homepage~\cite{apache:hadoop}.

The Apache™ Hadoop™ project develops open-source software for reliable,
scalable, distributed computing.

The Apache Hadoop software library is a framework that allows for the
distributed processing of large data sets across clusters of computers using a
simple programming model. It is designed to scale up from single servers to
thousands of machines, each offering local computation and storage. Rather than
rely on hardware to deliver high-avaiability, the library itself is designed to
detect and handle failures at the application layer, so delivering a
highly-available service on top of a cluster of computers, each of which may
be prone to failures.

We mainly use the Hadoop MapReduce framework rather than HDFS.

\subsection{k-means}
There are many materials that introduce k-means
algorithms. MacQueen, J. et al. proposed k-means algorithm in
\cite{algo:kmeans1}. In \cite{algo:kmeans2} the author introduced a simple
k-means MapReduce algorithm. 

The dataset is relatively large comparing to the memory size. Thus we need a
mechanism to deal with the incompatibility of memory. MapReduce is a solution to
this problem. The detail of our algorithm is documented in the following
section.

\subsection{Dataset}
% TODO 

\section{Clustering Algorithm}
In this section we discuss the k-means clustering algorithm used in our
implementation. Note that to find proper initial clusters, which may lead to
fewer iterations and good result, we also implemented a Canopy clustering
MapReduce algorithm.
\subsection{Canopy Clustering}
Canopy method is a relative easy but efficient clustering method. The basic idea of Canopy is to put the similar items into the subset, which are called Canopies. Different Canopies can overlap each other, but there is no node, which doesn't belongs to any of Canopy. Thus, here we use this method to pretreat the dataset before Kmeans method.

The algorithm is built up by the following MapReduce phases,
\begin{enumerate}
  \item Choose two distance threshold $t1$ and $t2$, where $t1>t1$.	
  \item Choose a node P from the list, and quickly calculate the distance
  between node P and Canopy, if this distance is less than t1, put P into this
  Canopy, if no such kind Canopy exits, take P as a new Canopy.
  \item if P was near enough to some Canopy, actually it means the distance
  between P and the Canopy is less than t2, then delete node P from the list.
  \item repeat 2 and 3 until the list is empty.
\end{enumerate}
The parallel strategy described as map-reduce is to produce some Canopy in the
mapper work and get the final Canopy set in the reducer work. The more details
are described as follows

\begin{enumerate}
    \item Mapper: \verb|<LongWritable, Text>|$\rightarrow$
    \verb|<"centroid",CanopyCluster>|. It reads the input file content as
    value. The key is a fixed string "centroid", and the value is the center
    node of the Canopy.
    \item Combiner: which can be considered as a local reducer.
    \item Reducer: recalculate the center point of local Canopy and update the
    information of the global Canopy. Write \verb|<identifier,CanopyCluster>| into context.
\end{enumerate}

\subsection{k-means Clustering}
Mahout Project~\cite{apache:mahout} is a data mining framework under Apache™
Fundation. It contains a k-means implementation. The blog~\cite{algo:kmeans3}
gives a very detailed view of the algorithm. Our algorithm mainly based on the
idea of Mahout Project. Actually, this algorithm is very similar to the BFR
algorithm in our lectures.

The algorithm is built up by the following MapReduce phases,
\begin{enumerate}
  \item The k-means iteration, output every cluster centroid if
  converged.  	
  \item Assign every point to a known cluster and output the result.
\end{enumerate}

The detail of the first phase is described as follows
\subsubsection{Iterations}
\begin{enumerate}
  \item Mapper\\\verb|<LongWritable, Text>|$\rightarrow$
  \verb|<LongWritable,KmeansCluster>|. It reads the input file content as value.
  The key is the value offset in the file. Then it convert the value to a
  VectorDoubleWritable, which is used to represent the feature vector. It finds
  the nearest cluster to the vector, and output cluster id as key, a new cluster
  containing only point as value.
  \item Combiner\\\verb|<LongWritable, KmeansCluster>|$\rightarrow$
  \verb|<LongWritable, KmeansCluster>|. It reads the output from Mapper and
  combine those tuples who have the same cluster id(meaning that they are
  assigned to the same cluster) using KmeansCluster.omitCluster(). This function
  reduced the network transmission flow because the actual meaningful
  information need to be communicated between different nodes are only the N,
  SUM and SUMSQ of clusters, which is described in the lecture slides of BFR algorithm.
  \item Reducer\\\verb|<LongWritable, KmeansCluster>|$\rightarrow$
  \verb|<LongWritable, KmeansCluster>|. It reads the cluster id as key and the
  KmeansCluster as value. It adds the N, SUM and SUMSQ. The result leads to the
  combination of clusters. Finally the reducer outputs the result clusters of
  this single iteration. This clusters are input of next iteration. During
  reducer, if the movement of one cluster if less than a threshold, it's said
  to be ``Converged''. If a cluster is converged, a counter in context will
  increase by one.
  \item If, in the driver, the counter equals the number of clusters, meaning
  that all clusters are converged, this phase is finished.
\end{enumerate} 
\subsubsection{Assign Point to Clusters}
After the iterations, the clusters are stable. Next we are going to assign every
point to its nearest cluster. This phase only require mapper to do all the
works, since this procedure is highly parallelized. Each two point are
independent of each other.

The mapper takes the following form:\\
\verb|<LongWritable, Text>|$\rightarrow$\verb|<Text, Text>|\\
It takes the data offset in file as key and the data as value. First it convert
the Text object into a VectorDoubleWritble object that represents a data point.
For each point we find the nearest cluster and output the cluster id and some
other useful information for statistic and evaluation as value.


\section{Evaluation}
In our experiment, we calculate the purity to evaluate the results. A higher purity means a more accurate clustering result.
\subsection{Evaluation Algorithm}
Because we don't know the map relationship between our clusters and the GrandTruth classes, so we must find a maximum match from the clusters to origin classes, so that we can get a maximum purity.
The basic idea to find such a maximum match is to find a highest purity of the clusters for each class, and map the cluster to the origin class.
The detail of the Algorithm is described as follows
\begin{enumerate}
  \item For each origin class, find the cluster-ID, which has the maximum purity.  	
  \item Find the class which has the maximum purity among the maximum purity of different classes. And assign the maximum cluster-ID to this class.
  \item Set all current purity in this class to zero, and set all the cluster-ID in other classes to zero.
  \item Repeat 1,2,3 until all classes have an assigned cluster-ID.
  \item Output the purity for each classes.
\end{enumerate}
\subsection{Evaluation Results}
%to-do

\bibliographystyle{plain}
\bibliography{myrefs}

\end{document}
